\clearpage
\phantomsection
\addcontentsline{toc}{chapter}{The Shika na Mikono Teaching Philosophy}
\chapter*{The Shika na Mikono Teaching \hfill \\ Philosophy}

%\clearpage
%\phantomsection
%\addcontentsline{toc}{section}{The Value of an Interactive Math and Science Education}
\section*{The Value of an Interactive Math and Science Education}

Math and science education provides a major challenge throughout the world. In fact, formal education as a whole is subject to a large number of constraints which limit a student's educational environment and may hinder her overall capacity for understanding and thinking critically.

As human beings, we do not live merely to assimilate to our surroundings and mimic what is being done by others. Rather, it is our unique advantage to continually challenge ourselves to extend our current limits of understanding. Utilization of this remarkable opportunity as a global community requires individuals with a scientific mindset who are able to ask critical questions and invent solutions to the increasingly difficult challenges facing a sustainable coexistence with nature.

It is the aim of education to immerse developing minds in an environment that fosters these kinds of analytical and creative abilities. Unfortunately, such goals often fail to be realized due to a number of limitations – time, money, teacher-to-student ratio and educational resources, to name a few. Although these challenges can put many communities at a great disadvantage and create divides among learners, they should not prevent educators from inspiring imagination and innovativeness among their students. The materials we need are already available in our villages and in our towns. The key ingredients in science education are not precision glassware, imported reagents, or massive loans. The key ingredients are curiosity, creativity, and the ability of teachers to use what they already have to provide students with experiences that broaden their understanding of the world. When students explore the world with readily available materials instead of expensive lab apparatus and high-end chemicals -- when they see parts of their own world appear in the classroom -- they gain an understanding that bridges scientific theory and daily life. That understanding sheds light on the world beyond the laboratory, which lets them wield scientific thinking anywhere.

While this goal can be achieved by anyone regardless of location, it does require a suitable learning environment. Training students to follow commands and adhere to a linear method of learning can only produce obedience, not intelligence. In order to continue to adapt and learn in a quickly changing world, students must be able to interact with their surroundings and engage themselves in the content of their studies.

No fields offer such readily available opportunities to practice these skills as do math and science. The ability to observe and appreciate the applications of math and science in our daily lives is an essential component of a person's intellectual development. It is through interactive education that this can be acquired.

A hands-on, interactive approach to learning truly has many benefits inherent in and of itself. It fosters a deeper understanding through experiential discovery as opposed to rote memorization. It encourages learning through enjoyment and excitement, which are much stronger motivators than the fear of punishment. It allows for the development of creativity and curiosity within students, invaluable qualities with benefits far beyond the realm of formal education. It empowers students to seek their own answers and truth in all matters -- not just for test questions in school, but for the ones that really matter in life.

Rather than merely encouraging an interactive approach to learning, the aim of \emph{Shika na Mikono} is also to illuminate the relationship between the classroom and all the aspects of a student's life and natural environment. Thus, science becomes the study of reality. An understanding and appreciation of the world around us is what allows us as human beings to interact with and respect all other things in our universe. It is in this way that students can become equipped to take on new challenges and ensure the prosperity of generations to come.

\vfill
\pagebreak

%\clearpage
%\phantomsection
%\addcontentsline{toc}{section}{The Teacher's Role in Student Development}
\section*{The Teacher's Role in Student Development}

To learn science, students must interact with the world around them. They must ask their own questions and seek their own answers. They must see things and they must grasp them in their hands; hence the name of this book: \textit{shika na mikono} is Swahili for ``grasping in hands.''

It is our fervent belief that every student in the world should perform science practical exercises. For too long we have heard complaints that schools lack the materials necessary for these exercises. This book attempts to make clear that students may perform science practicals at any school, most especially at those without traditional laboratories, starting today. Everything teachers need to create these hands-on learning experiences is available locally and/or at low cost.

Many national syllabi require practical exercises, often on their national examinations. This is good. Critically, however, we urge teachers to expand the scope of students' hands-on work beyond the practicals for the national exams. Every topic, every lesson may be a ``practical'' -- not just a demonstration on the front bench but an opportunity for students to touch and manipulate and discover on their own.

In this vision, the science teacher becomes a guide, someone who can assemble parts of the natural world into a compelling lesson and ask the questions that help students see how things work. In this capacity, the science teacher remains a resource irreplaceable by the march of technology. Photocopy machines produce student editions of notes much more efficiently than teachers copying them to the board for students to copy again. Instructional films shown on low-cost solar-powered projectors offer students articulate explanations and demonstrations. But no technology can replace the essential role of the modern science teacher: s/he is an architect, one who builds a space in which students can learn for themselves, and a shepherd, who tends to their learning through that discovery.

The aim of this book is to inspire and empower this sort of teaching. For years, many educators have bantered about the phrase ``student-centered teaching.'' This sounds rather like patient-centered medicine -- anything else is simply absurd. The focus of a lesson must always be the experience of the student. To prepare such a lesson, the teacher should answer the following questions:
\begin{itemize*}
\item What will the student do in class? 
\item How will she use her hands to interact with the world? 
\item What will the student observe with her own senses? 
\item Given these experiences, what questions will arise from the student's observations? 
\item Given these questions, how might the teacher respond to provoke further inquiry and critical thinking? 
\item How might the student's peers respond to build a common understanding? 
\item How might the student, through further observation and experimentation, arrive at the answer herself? 
\item Given these goals, what experiences will put her on the journey to that answer? 
\item What series of activities should be offered to her to facilitate that discovery? 
\end{itemize*}

This is student-centered teaching -- a lesson plan crafted around the experience of the student -- the internal, cognitive, and emotional experience of being in class that day.

The process of answering these questions involves several steps. The teacher must first organize the material that the student is to understand into a well-structured framework: logical, sequential, and hierarchical. Then, using this framework, the teacher should design activities for the student to discover each aspect of the material. These activities should be sequenced to expand understanding, moving from simple phenomena to the more complex; from the specific to the general. Discussion questions should seek first to uncover core phenomena and then to link each new insight with what the students already understand about their world. Targeted questions catalyze introspection, group discussions, and the realizations necessary for the students themselves to start articulating scientific theories. Once the students have discovered phenomena, linked them to pre-existing understanding, and begun articulating general theory, the teacher can help focus and form these articulations into the accepted vocabulary and nomenclature of modern science.

Finally, we teachers must embrace questions; we must encourage students to ask about what they do not understand. Rather than answer these questions directly, whenever possible we should design experiments or ask questions in return that allow students to find answers for themselves. As role models, we also must embrace the limits of our own understanding. Often students ask questions to which we do not know the answer. This is a fundamental aspect of science education. Our job is to help students to understand the world better, to guide them in that discovery. Our job is not to know everything; this is neither necessary, nor is it possible, nor even desirable. When our students observe us confronting the unknown, when they see how we ask questions and perform experiments ourselves to seek out the truth, then they become more comfortable asking questions and seeking answers themselves. This experience helps them to understand the true power of science, that a person anywhere may always find the answer.

Let us gather the world around us and put it in the hands of our students, so they might understand how it works. Let us let them grasp it in their hands -- \textit{walishike na mikono yao.}
