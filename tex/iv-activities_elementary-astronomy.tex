\section{Elementary Astronomy}


%Solar System

%\subsection{Solar System Mobile}
%\begin{itemize}
%\item{Preparation time: a few hours}
%\item{Materials: flour, water, balloons, mixing bowl, newspaper or old papers, string, sticks}
%\item{Construction: Blow up the balloons, one for each of the 8 planets and sun. Make the paper mache mixture with flour and water; you want a watery-glue texture. Wet the paper in this mixture and apply artistically to the balloons until you have a layer a couple papers-thick on each balloon. Leave the balloon slightly exposed at the bottom. When the papers are dried, pop the balloons within and set to work making them look like planets. Use paint, markers, or colored pencils. Attach string and hang them as a mobile. If you want to get fancy, you can place the string between layers of paper before it dries, thus saving yourself some tape or glue.}
%\item{Theory: This activity is helpful to explain to students what is actually happening off the world. This mobile is helpful to remember that there is more to the solar system than just earth.}
%\end{itemize}


\subsection{Solar System Mobile}

\subsubsection*{Learning Objectives}
\begin{itemize}
\item{To understand the arrangement of celestial bodies in our solar system}
\end{itemize}

\subsubsection*{Background Information}
A solar system is the group of bodies that surround a star. In our solar system we have 8 planets as well as comets, asteroids and asteroid belts, dwarf planets, and many moons. All of these objects move in ellipses around our star, the Sun. There are too many objects in our solar system to count, but we have names for the largest of these: the planets, their moons and some of the comets and asteroids. Outward from the sun, the planets are Mercury, Venus, Earth, Mars, Jupiter, Saturn, Uranus and Neptune. Pluto was once a planet but has since been reclassified a dwarf planet. Each planet has a unique size, environment and history, though Earth is the only planet that we know of with life.

\subsubsection*{Materials} Flour, water, balloons, mixing bowl, newspaper or old papers, string, sticks

\subsubsection*{Preparation Procedure}
\begin{enumerate}
\item{Blow up nine balloons, one for each of the 8 planets and sun.}
\item{Make the paper mache mixture with flour and water; you want a watery glue texture.}
\item{Wet the paper in this mixture and apply artistically to the balloons until you have a layer a couple papers-thick on each balloon.}
\item{Leave each balloon slightly exposed at the bottom.}
\item{When the papers are dried, pop the balloons inside.}
\item{Use paint or marker pens to make the paper balls look like planets.}
\item{Attach string to each of the planets.}
\item{Hang the planets on sticks with the sun in the middle and each of the planets at different points moving away from the sun.}
\end{enumerate}

\subsubsection*{Activity Procedure}
\begin{enumerate}
\item{Hang the mobile from a beam or ceiling so that the planets are suspended around the sun at different distances.}
\item{Identify each of the planets and discuss them.}
\end{enumerate}

\subsubsection*{Cleanup Procedure}
\begin{enumerate}
\item{Dispose of any remaining liquid before it dries. Do not pour it down a sink because it might clog.}
\item{Cap any markers or paint and return all materials to their proper places.}
\end{enumerate}

\subsubsection*{Discussion Questions}
\begin{enumerate}
\item{Name the eight planets.}
\item{Which planet is the largest? Is it solid or gas?}
\item{Which planet is the smallest? Is it solid or gas?}
\item{What do the first four planets have in common?}
\item{What do the last five planets have in common?}
\end{enumerate}

\subsubsection*{Notes}
This activity is helpful to explain to students what is actually
happening outside of the world. This model reminds students that there are other objects in the solar system besides Earth. Remember that the planets are all at different distance from the sun, but they are all in the same plane. For this reason, hang the planets at about the same height.



%Constellations

%\subsection{Star Gazing}
%\begin{itemize}
%\item{Preparation time: 0 minutes}
%\item{Materials: none}
%\item{Procedure: Take the students out at night. Look for constellations, starts, planets, and even satellites. In addition to planets, look for Orion’s Belt, the Southern Cross, and more constellations. Due to the large amount of information regarding this topic, we cannot include this information. However, a quick internet search will give you what you need.}
%\item{Theory: For the longest time, stargazing was some of the most important aspects of navigation and even religion. Recreate this experience by finding stars and constellations. Tell the stories behind them, and encourage students to find their own constellations and give their own stories.}
%\end{itemize}


\subsection{Star Gazing}

\subsubsection*{Learning Objectives}
\begin{itemize}
\item{To identify objects in the night sky}
\item{To understand various structures and bodies in the galaxy and universe in relation to the earth}
\end{itemize}

\subsubsection*{Background Information}
Astronomy is one of the oldest sciences. It has been used for thousands of years in navigation and has provided the proof or evidence for many laws and theories of nature such as gravitation and relativity.

\subsubsection*{Activity Procedure}
\begin{enumerate}
\item{Take the students out at night where there is little light from lamps and fires.}
\item{Look for constellations, stars, planets and satellites. Discuss the reason for having constellations and the motion of the sky over the course of a night and a year.}
\end{enumerate}

\subsubsection*{Results and Conclusions}
Especially in rural areas, the stars and other celestial bodies are very clear. Depending on the time of year, different planets and constellations will be visible. The most obvious constellations are Orion, Ursa Major and the Southern Cross. The brightest star is Sirius. If the sky is clear then our galaxy, the Milky Way, is visible as a bright stripe across the sky.

\subsubsection*{Discussion Questions}
\begin{enumerate}
\item{What objects did you observe in the night sky?}
\item{What are the brightest objects?}
\item{Is it easier to see the stars when the moon is present, or when it is absent?}
\item{What constellation could you use to know which direction is North?}
\end{enumerate}

\subsubsection*{Notes}
Before going outside at night, check a star chart and planet charts to find out which objects should be visible that night. This will make it easier to identify objects and help students to see them. If you have binoculars, they can be used to see planets easily or to see the difference between a single star system and a binary star system, or between a star and a distant galaxy. If possible, tell students the stories behind the constellations and see if they can make their own constellations.




%Moon / Tides
