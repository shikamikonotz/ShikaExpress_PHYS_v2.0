\section{Work, Energy and Power}

%Work


\subsection{Work as Heat, Part A}
\begin{itemize}
\item{Preparation time: 5 minutes}
\item{Materials: thin strip of metal, pliers}
\item{Procedure: Take a piece of metal. Use a set of pliers to bend the metal back and forth. Feel the temperature of the metal.}
\item{Theory: Work can manifest itself in a variety of ways. One of the most common ways is the rise in temperature. By moving the metal back and forth, you are doing work on the metal. This work is converted into heat. This heat is evidenced by the rise in temperature in the metal. }
\end{itemize}

\subsection{Work as Heat, Part B}
\begin{itemize}
\item{Preparation time: 0 minutes}
\item{Materials: radio antennas, old or new }
\item{Procedure: The radio antennas operate in a telescopic motion. Pull the radio antenna in and out for one full minute. Do not break the antenna in this movement. Observe the temperature of the antenna after the work is over.}
\item{Theory: Again, you are doing work on the radio antenna by moving it in an out quickly. Through this action, the antenna heats up. This is the evidence of the work you have been doing. Work is defined as force times distance or. In this case, the force is the effort required to move the antenna in and out while the distance is the length of the antenna.}
\end{itemize}

\subsection{Work as Light}
\begin{itemize}
\item{Preparation time: 0 minutes}
\item{Materials: duct tape, or other tape that holds together tightly.}
\item{Procedure: Cut two pieces of duct tape. Press the ends of the bottom pieces of tape together but allow the top pieces of tape to be apart. Hold tightly to both pieces of tape at the top, and quickly rip them apart. Observe the blue light when the tape comes apart.}
\item{Theory: Pulling the tape apart quickly creates a faint blue light. It is best to observe this light at night since it is so faint. In this activity, this is the work being done to pull the tape apart. Unlike the previous activities, this work is released as light. This phenomenon as where work manifests itself as light is called triboluminescence. This is the same phenomenon that causes the green light when snapping wintergreen mints.}
\end{itemize}


%Energy

	%Kinetic Energy, Potential Energy
	
	%Transformation of Energy
	
	%Conservation of Energy
	
	
%\subsection{Pencil Launcher}
%\begin{itemize}
%\item{Preparation Time: 5 minutes}
%\item{Materials: Clothes clip, thread, two pencils}
%\item{Procedure: Open the clip and tie the closed end with thread so that the clip stays open against the tension of the spring. Place the clip flat on a table and place two pencils next to the clip, one on either side, so that the eraser touches the tied end and the tips point out in opposite directions along the table. Cut the thread holding the clip open and stay clear of the flying pencils.}
%\item{Theory: The spring inside the clip holds energy when it is forced to contract. When the clip is allowed to close, the potential energy of the spring is transformed into mechanical energy as the clip moves, forcing the pencils away at a decent speed.}
%\end{itemize}


\subsection{Potential Energy of a Spring}

\subsubsection*{Learning Objectives}
\begin{itemize}
\item{To observe the change in energy from potential to kinetic}
\end{itemize}

\subsubsection*{Background Information}
In a closed system, where no force acts on the objects, the total energy remains constant.  In the case of mechanical energy, this means that potential energy and kinetic energy can change, but their total remains the same.  Springs and other elastic materials also have potential energy in the form of elastic potential.

\subsubsection{Materials} 
Clothes pin, thread, two pencils

\begin{figure}[h]
\begin{center}
\def\svgwidth{100pt}
\input{./img/energy-conservation.pdf_tex}
\caption{Energy conservation in a spring}
\label{fig:energy-conservation}
\end{center}
\end{figure}

\subsubsection{Activity Procedure}
\begin{enumerate}
\item{A clothes pin has two ends: one for gripping clothes and one for pushing with fingers. Press the finger ends together so the gripping end is open as far as possible. Use a thread to tie the finger ends together so that the gripping ends stay wide open.}
\item{Place the clothes pin side-down on a table and place two pencils touching the clothes pins as shown in the figure below.}
\item{Cut the thread holding the clothes pin open, the clothes pin should open quickly and push the pencils away.}
\end{enumerate}

\subsubsection*{Cleanup Procedure}
Return all materials to their proper places.

\subsubsection*{Discussion Questions}
\begin{enumerate}
\item{What two types of energy are being used here?}
\item{Describe the change in energy occurring here.}
\end{enumerate}
 
\subsubsection{Notes}
The spring inside the clip holds energy when it is forced to contract.  
When the clip is allowed to close, the potential energy of the spring
is converted into mechanical energy as the clip moves, forcing the pencils
away quickly.

The interconversion of potential and kinetic energy may also be shown with a simple pendulum. Another option is to hold a heavy book above the table and then to drop it on the table.


%Power