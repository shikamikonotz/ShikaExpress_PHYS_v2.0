\section{Static Electricity}

%Methods of Charging (friction, contact, induction)

\subsection{Salt and Pepper Trick}
\begin{itemize}
\item{Preparation Time: 1 minute}
\item{Materials: salt, pepper flakes, pen, dish}
\item{Procedure: Mix a spoonful of salt with a spoonful of pepper and place it on a piece of paper or dish. Charge the pen by rubbing it on your hair or a piece of cloth and hold it over the salt and pepper. Which flakes jump to the pen?}
\item{Theory: Both salt and pepper will be attracted to the pen, but the salt is too heavy to move so only the pepper will make the jump.}
\end{itemize}


\subsection{Concept of Static Electricity}

\subsubsection*{Learning Objectives}
\begin{itemize}
\item{To demonstrate the charging of an object} 
\item{To explain the concept of static electricity} 
\end{itemize}

\subsubsection*{Background Information}
All objects carry charges which attract or repel other charges. Normally, the total charge on an object is zero, meaning an equal number of positive and negative charges. Sometimes, however, charges can be added to or removed from an object, causing the object to become charged. This can be seen when two objects attract or repel each other. 

\subsubsection*{Materials}
Sweater wool, dry plastic sheets

\subsubsection*{Activity Procedure}
\begin{enumerate}
\item{Place two transparent plastic sheets together on top of a book.}
\item{Rub the two plastic sheets with the sweater wool several times.} 
\item{Separate the two plastic sheets slowly and listen to what happens.} 
\item{Bring the plastic sheets closer and observe what happens.} 
\item{Separate the two plastic sheets and rub them individually with the sweater wool several times.} 
\item{Bring the sheets close to one another and observe what happens.} 
\end{enumerate}

\subsubsection*{Results and Conclusions}
When the two plastic sheets are separated a cracking sound is heard. When they are brought together they are attracted to each other. This is because the upper sheet acquires a different charge from the lower sheet. 
When the two plastic sheets are rubbed separately and brought closer, they tend to repel. This is because they acquire the same charge. 
Charging by rubbing removes electrons or adds electrons. When rubbing the plastic sheets with a sweater, electrons will move from the sweater to the upper plastic and the protons from the lower plastic will attract the electrons from the upper one. That is why the sheets attract when the they are rubbed together. When rubbing the sheets separately, you give them both a negative charge (electrons) from the sweater. That is why they repel each other. 

\subsubsection*{Clean Up Procedure}
Collect all the used materials, storing items that will be used later.

\subsubsection*{Discussion Questions}
\begin{enumerate}
\item{Why do we hear a crackling sound when we pull a sweater over our head?}
\item{Why do the sheets attract each other in one case but repel each other in another case?}
\end{enumerate}

\subsubsection*{Notes}
The crackling sound heard when the two sheets separate is caused by the movement of charges. The concept is the same as that of lightning, which involves moving charges. As the charges move, they displace air and leave a trail of low pressure. Air then rushes in to fill the empty space or low pressure, causing a crack sound. 


\subsection{Electrostatics}
\begin{itemize}
\item{Preparation Time: 5 minutes}
\item{Materials: Plastic ruler and piece of nylon cloth, a glass object and silk cloth, or a latex balloon and piece of fur (or hair), small pieces of metal foil, thread}
\item{Procedure: Rub the plastic ruler against the piece of nylon cloth. This transfers electrons between the two items, producing an electrostatic charge. If the piece of nylon cloth is small, try suspending it from a thread near the ruler. As the two items have opposite charges, they attract each other, causing the nylon to lean towards the ruler.\\
Crumple a piece of foil into a small ball, and suspend it from a thread. Bring the charged ruler near to the foil ball. The charge on the ruler should cause an induced dipole in the foil, which is in turn attracted to the charge on the ruler, causing the foil to lean towards the ruler.\\
If you rub the ruler on two different small pieces of cloth, try suspending the two cloths near each other. As they have the same charge, they will repel and lean away from each other.\\
N.B.: The above can be performed by rubbing a plastic ruler on nylon cloth, or by rubbing glass on silk, or by rubbing latex on fur. Some clothing is made out of nylon. Silk is commonly found in the liner to suit jackets. Other combinations of items can also produce static electric charges. It is best to try these on your own before showing them in front of class.\\
This demonstration is best performed in a room with no wind or air currents, which will make it difficult to see the objects leaning towards each other. The static charges will last for a longer time if there is low humidity and a low amount of dust. On humid or dusty days, the static charges will discharge faster. This is a good alternative to the Gold Leaf Electroscope, which is rather expensive and unnecessary.}
\end{itemize}

%Electroscope (detection of charges)

\subsection{Construction of an Electroscope}

\subsubsection*{Learning Objectives}
\begin{itemize}
\item{To construct a leaf electroscope} 
\item{To understand the mode of action of an electroscope} 
\end{itemize}

\subsubsection*{Background Information}
A gold-leaf electroscope is used to detect the presence of electric charge on an object. It consists of a conductor attached to a very thin leaf of gold. When a charged object is brought to the electroscope, the charge moves along the conductor and leaf, causing them to repel each other. If the leaf deflects it means that there is a charge on the electroscope and therefore on the object being measured. If the leaf does not deflect, it means that there is no charge present. 

\subsubsection*{Materials}
Clear jar with a plastic cap, iron nail, small piece of aluminium foil, glue

\begin{figure}[h!]
\begin{center}
\def\svgwidth{200pt}
\input{./img/al-leaf-electroscope.pdf_tex}
\caption{Aluminium foil electroscope}
\label{fig:al-leaf-electroscope}
\end{center}
\end{figure}

\subsubsection*{Preparation Procedure}
\begin{enumerate}
\item{Use the iron nail to make a hole in the plastic cap.} 
\item{Insert the nail down into the hole so that only about 1 cm is sticking up above the cap.} 
\item{Use glue to secure the nail in the cap.} 
\item{Cut a piece of aluminium foil 0.5~cm by 2~cm.} 
\item{Attach one end of the foil (only the tip) to the iron nail with glue about 2~cm from the bottom of the nail.} 
\item{Bend the foil at its pivot (where the glue is) so make sure that it can swing easily. If it cannot, wrap the end of the foil around a piece of thin wire and glue the wire to the nail so that the foil is free to swing close to the nail.} 
\item{Close the cap with the nail and foil on the jar.} 
\end{enumerate}

\subsubsection*{Activity Procedure}
\begin{enumerate}
\item{Bring a charged object near the top of the nail.} 
\item{Observe any deflection on the foil leaf.} 
\end{enumerate}

\subsubsection*{Results and Conclusions}
When a charged object is brought close to the nail, the foil leaf deflects from the nail. This is because the charged object repels one type of charge in the nail and attracts the opposite charge. This causes one type of charge to move down the nail into the bottom of the nail and into the leaf. The charges in the leaf and nail repel each other, so the leaf deflects away from the nail. 

\subsubsection*{Clean Up Procedure}
Return all materials to their proper places. Store the electroscope for later use.

\subsubsection*{Discussion Questions}
\begin{enumerate}
\item{Why is it necessary to use a metal nail instead of plastic or wood?}
\item{Why do we close the leaf in the jar?}
\item{What does the foil leaf do when a charged object is brought near the nail head?}
\end{enumerate}

\subsubsection*{Notes}
A gold-leaf electroscope works as described above. Gold is used because it can be made very thin, so it has a very small weight and can deflect easily. Aluminium is not as thin so it cannot deflect as much, but it still shows the presence of charge. 


\subsection{Detection of Charges}

\subsubsection*{Learning Objectives}
\begin{itemize}
\item{To determine the presence of charges} 
\item{To demonstrate the charging of the leaf electroscope} 
\end{itemize}

\subsubsection*{Background Information}
Charges form the basis of our understanding of static and current electricity. They are simply electrons; the presence of electrons increases the negative charge on an object, the absence of electrons decreases the negative charge on an object, and it is electrons which move through a conductor to create electric current. However, electrons are too small to see, even with a microscope. Therefore, we can only detect the presence of charges, not the charges themselves.

\subsubsection*{Materials}
Glass and silk or plastic pen and hair, leaf electroscope (see above activity to construct this)

\subsubsection*{Preparation Procedure}
Collect the necessary materials.

\subsubsection*{Activity Procedure}
\begin{enumerate}
\item{Take glass and rub it with the silk or take the plastic pen and rub it in your hair, then quickly bring it close to the nail of the electroscope.}
\item{While the glass is still touching the metal cap observe the deflection of the foil in the electroscope  (is there attraction or repulsion with the metal nail?).} 
\item{Record the results from your observation and determine if charge is present.} 
\item{Repeat these steps using notebook paper or any other objects you can find.} 
\item{Compare results from different materials.}
\end{enumerate}

\subsubsection*{Results and Conclusions}
By touching the charged body to the top of the metal cap, charges are transferred to the leaf of the electroscope. Because both are equally charged the leaf and conductor will repel each other showing the presence of charge whether negative or positive.
The amount of deflection is dependent on the amount of charge on the electroscope. An object with greater charge will cause the leaf to deflect more then an object with less charge. 

\subsubsection*{Clean Up Procedure}
Store items for later use.

\subsubsection*{Discussion Question}
From your experimental results, how can you know if there is charge on the object touching the electroscope?


%Capacitors


%Charge Distribution


%Lightning Rods