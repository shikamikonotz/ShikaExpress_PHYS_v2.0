% Unused Activities

%==================================================================================================%

\section*{Waves}


\subsection{Speed of Sound in Air}

\subsubsection*{Learning Objectives}
\begin{itemize}
\item{To understand the relationship between a wave's frequency, speed and wavelength} 
\item{To calculate the speed of sound in air using a resonance tube} 
\end{itemize}

\subsubsection*{Background Information}
A wave's speed or velocity is directly related to the wave's frequency and wavelength. If the wavelength and frequency can be determined, the speed can be calculated easily. We can use a resonance tube or sonometer to find the wavelength and frequency of a wave, therefore allowing us to calculate the speed of the sound wave in that medium.  

\subsubsection*{Materials}
Resonance tube (this can be made: see the activity about constructing a resonance tube in this book), tuning fork or wind instrument like a flute or recorder, water, metre rule

\subsubsection*{Hazards and Safety}
\begin{itemize}
\item{If you are using tuning forks, do not hit them on a table or any other hard object. Over time, this will damage the tuning forks and changes their frequencies until they are no longer useful. Instead, hold the middle of the handle and hit one of the fork's prongs on the sole of your shoe or any other hard rubber object.} 
\end{itemize}

\subsubsection*{Preparation Procedure}
\begin{enumerate}
\item{Set up the resonance tube with water.} 
\item{Place the metre rule next to the resonance tube so that the 0 cm mark is at the top of the resonance tube.} 
\end{enumerate}

\subsubsection*{Activity Procedure}
\begin{enumerate}
\item{Use a tuning fork or flute to make a single musical note.} 
\item{Place the fork or flute over the resonance tube and adjust the water level until the resonance can be heard in the tube. The note produced by the fork or flute will be heard at the top of the tube when the water is at a certain level.} 
\item{Record the distance between the water level and the top of the tube at the level when the water reaches a level where resonance can be heard.} 
\item{Also record the frequency of the note. If you are using tuning forks, the frequency is written on the handle. If you are using a flute, the frequency depends on the note you are playing. A table of musical notes and their frequencies can be found in any textbook. The easiest to use is A, which has a frequency of 440 Hz.} 
\item{Use your values of length (wavelength) and frequency to calculate the speed of sound in the tube.} 
\item{Repeat these steps for several notes and compare your values of speed of sound.} 
\end{enumerate}

\subsubsection*{Results and Conclusions}
It will be seen that the product of wavelength and frequency will be almost the same. This is because the speed of sound in air is constant (depending on humidity and density). Therefore, the product of wavelength and frequency must be constant. As frequency increases, the wavelength (the length of the tube above the water level) decreases.  

\subsubsection*{Clean Up Procedure}
\begin{enumerate}
\item{Empty the water from the resonance tube.} 
\item{Return all materials to their proper places.} 
\end{enumerate}

\subsubsection*{Discussion Questions}
\begin{enumerate}
\item{What was your average value for the speed of sound in air?}
\item{As the frequency of the musical note increased, did its wavelength increase or decrease?}
\item{If the speed of a wave remains constant, what is the relationship between wavelength and frequency of a wave?}
\end{enumerate}

\subsubsection*{Notes}
There is room for error in this experiment because you are trying to measure length as the water level is moving. Also, the flute or other instrument you are using to create a note may not be perfectly in tune and so may have a slightly different frequency. Do several experiments in order to find a consistent value for the wavelength.


\subsection{Sound Amplifier}

\subsubsection*{Learning Objectives}
\begin{enumerate}
\item{To understand the amplification of mechanical waves}
\item{To observe the amplification of sound in a hollow cavity}
\end{enumerate}

\subsubsection*{Background Information}
Everything can vibrate.  If a wave drives the vibration of another object or surrounding air, we say that the wave is amplified as its amplification has increases with the resonating body.  This principle is used when marimbas are played inside gourds.

\subsubsection*{Materials}
Plastic water bottle, string or thread, match or small stick

\subsubsection*{Preparation Procedure}
\begin{enumerate}
\item{Make a small hole in the bottom of the bottle.}
\item{String one end of the thread through the hole.}
\item{Tie the end on the inside of the bottle to the match or small stick so that it cannot be pulled back through the hole.}
\item{Leave the length of string hanging out of the bottom of the bottle}
\end{enumerate}

\subsubsection*{Activity Procedure}
\begin{enumerate}
\item{Pull the string taught and have a student hold the top of the bottle.}
\item{Pluck the string.}
\item{Try plucking just the string and then the string and bottle together.}
\item{Try plucking the string with the cap on or off.}
\item{Observe the various effects of the sound.}
\end{enumerate}

\subsubsection*{Results and Conclusions}
When the string is plucked by itself, the sound it creates is very small.  However, when the string is attached to the bottle, the sound is louder.  The vibration of the string causes the bottle itself to vibrate.  Rather than hearing just the sound of the string vibrating, we hear the sound of the bottle, which produces noticeably greater amplitude.

\subsubsection*{Cleanup Procedure}
Return all materials to their proper places.

\subsubsection*{Discussion Questions}
\begin{enumerate}
\item{What was the difference between the sound produced by the string and the sound produced by the string and bottle together?}
\item{What causes the sound you hear to be louder?}
\item{What was the difference in sound between using the cap and not using the cap?}
\end{enumerate}

\subsubsection*{Notes}
This effect can be difficult to detect if the bottle is small or if the frequency of the string is much higher or lower than that of the bottle.  Vary the length of the string until you get clear resonance.


\subsection{Sound in a Medium}
\begin{itemize}
\item{Preparation Time: half hour}
\item{Materials: Large jar with lid, glue, bicycle pump needle, string, cell phone, vacuum pump (see Reverse Pump)}
\item{Procedure: Poke a small hole in the jar lid and insert the pump needle with at least 1 cm above the lid. Secure the needle with glue, rubber, whatever you need to ensure that it is airtight. Program your cell phone to play something repeatedly at full volume. Hang the phone by the string in the jar so that it is not touching the sides; close the lid on the jar (if the glue is dry) and listen for the phone. You should still be able to hear the phone. Attach the vacuum pump from Reverse Air Pump to the needle on top of the jar and start pumping out the air. You should hear the sound of the phone decrease until it is not heard at all.}
\item{Theory: Sound requires a medium to travel. The denser the medium, the faster sound will travel. Without a medium, there is nothing to vibrate and therefore no sound. By removing the air in the jar, you are removing any material medium and the sound will not be able to travel beyond the cell phone speaker itself.}
\end{itemize}


\subsection{Musical Rubber Strip}
\begin{itemize}
\item{Preparation time: none}
\item{Materials: a length of rubber strip}
\item{Procedure: Stretch the rubber strip taught and pluck it. It should produce a musical note. Demonstrate that increasing the tension but keeping the length the same gives a higher note. Demonstrate that keeping the tension the same but increasing the length gives a lower note. Allude to tuning a guitar, which many students will have seen in church.}
\end{itemize}


\subsection{Barton’s Pendulums}
\begin{itemize}
\item{Preparation time: 5 minutes}
\item{Materials: Several pieces of string, one large weight (approximately 0.5kg), several small weights}
\item{Construction: Suspend a piece of string horizontally between two fixed objects. Hang the various weights from different points along the string. Each of the small weights should hang from a string of different length. The large weight should hang from a string of similar length to one of the small weights.}
\item{Procedure: Start the large weight swinging. Tell the students to take note of how this affects the behavior of the smaller weights. You should find that the small weight hanging from a string of the same length as the large one exhibits the largest oscillation.}
\item{Theory: The large weight acts as a driving force. Each small weight can swing as a simple harmonic oscillator. We know that a driving force will have the largest effect on a simple harmonic oscillator if the driving force is operating at the natural frequency of the oscillator. When the lengths of the two pendulums are the same, their frequencies are the same. You should be able to get “harmonics” going if you measure the lengths accurately (see string instruments).}
\end{itemize}


\subsection{Transverse Waves on a String}
\begin{itemize}
\item{Preparation time: depends, but in any case a long time}
\item{Materials: Show a design to a fabricator/welder and let them decide this. You can supply thin string, a small pulley, and a weight.}
\item{Construction: Using whatever driving device available. (I used a bicycle, like the men who pedal a bike wheel to drive a grinder), drive a piston with a very small amplitude (1 mm is fine). Whomever you find to do this will have their own way of doing this, but the easiest thing to do is just an offset axle, where the axle being driven jogs to one side a small amount. When you have a piston which can be driven at a very small amplitude by a bike wheel, car motor, etc., attach a string to the top of the piston and hang the other end of the string over a pulley about two meters (varies) away, suspended by a weight. Now you have a string that is driven at whatever frequency you choose.}
\item{Procedure: Pedal the bicycle or turn on the motor and increase the speed (frequency) until you see the fundamental on the string, a standing wave with one antinode and two nodes – the ends of the string. Chat about that for a minute, then increase the frequency until you get the first harmonic, then the second harmonic, etc., until you run out of juice in one way or another.}
\item{Variation: Drive the string with a speaker connected to a single-tone generator. This could be a simple circuit, in fact, allowing you to combine two of the biggest physics topics ever! Use a rheostat to vary the frequency of the circuit, ergo the speaker.}
\item{Theory: Every string has a natural frequency at which it will vibrate with ease, meaning with the greatest possible amplitude. This is called the fundamental (and is directly related to the fundamental as known in music theory, since all harmonics which follow are the octave, 5th, 4th, 3rd, etc.) and is the simplest standing wave. Doubling the frequency will give you the 1st harmonic (octave), which is the next simplest standing wave. All harmonics which follow are closer in frequency and become gradually more complex, but might be difficult to do on this machine unless you have a super-high gear ratio on the bike wheel or a speedy car motor.}
\end{itemize}

%==================================================================================================%